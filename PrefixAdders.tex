\section{Προθεματική Αθροιστές}
Στην ενότητα αυτή θα παρουσιαστεί μια νέα υλοποίηση αθροιστών, οι αθροιστές προθέματος,
ο οποίοι έλαβαν σημαντικό ρόλο στην επιτάχυνση καθώς και μείωση του συνολικού εμβαδού 
των αθροιστών. 


\subsection{Πρόβλημα προθέματος}

Ένα prefix problem ή πρόβλημα προθέματος 
ορίζεται από n εξόδους y ( $y_{n-1},y_{n-2}, ... ,y_0 $ ) , n εισόδους x ( $x_{n-1},x_{n-1}, ... ,x_0 $ ) και τον τελεστή $\circledast$ .
Κάθε έξοδος y υπολογίζεται με τον παρακάτω τρόπο :
\begin{equation}
\begin{split}
y_0 &= x_0\\
y_1 &= x_1 \circledast x_0\\
y_2 &= x_2 \circledast x_1 \circledast x_0\\
&...\\
y_{n-1} &= x_{n-1} \circledast x_{n-2} \circledast ... \circledast x_{1} \circledast x_0\\
\end{split}
\end{equation}

Επίσης μπορούμε να το εκφράσουμε και αναδρομικά :
\begin{equation}
\begin{split}
y_0 &= x_0\\
y_{i} &= x_{i} \circledast y_{i-1}\\
\end{split}
\end{equation}

Ένα απλό παράδειγμα προβλημάτων που αντιμετωπίζονται ως προβλήματα προθέματος είναι 
η πρόσθεση πολλών αριθμών. Έστω πως έχουμε ένα σύνολο μεγέθους n από αριθμούς 
$(x_{n-1},x_{n-2}, ... ,x_1,x_0)$, σύμφωνα με τον ορισμό που δόθηκε παραπάνω
χρειαζόμαστε ένα ακόμα σύνολο ίδιου μεγέθους $(y_{n-1},y_{n-2}, ... ,y_1,y_0)$ 
όπου κάθε στοιχείο του συνόλου αυτού υπολογίζεται αναδρομικά
\begin{equation*}
\begin{split}
y_0 &= x_0\\
y_{i} &= x_{i} + y_{i-1}\\
\end{split}
\end{equation*}
και το τελικό αποτέλεσμα είναι καταχωρημένο στο $y_n-1$.

Το πρόβλημα υπολογισμού κρατουμένου μπορούμε να το μετατρέψουμε σε
prefix problem δημιουργώντας το ζεύγος $(G,P)$ και αναθέτοντας στον τελεστή
$\circledast$ την παρακάτω λειτουργιά :
\begin{equation}
\begin{split}
(g_i,p_i) \circledast (g_k,p_k) &= (g_i + p_ig_k , p_ip_k)\\
(G_i,P_i) \circledast (G_k,P_k) &= (G_i + P_iG_k , P_iP_k)
\end{split}
\end{equation}

Με αυτόν τον τρόπο μπορούμε να υπολογίσουμε κάθε ενδιάμεσο κρατούμενο $c_i$
καθώς και το κρατούμενο εξόδου $c_n$ για έναν αθροιστή των n-bits όπου $c_i = G_i$
και για $n \geq i \geq 0$ έχουμε 
\begin{equation}
\begin{split}
(G_0,P_0) &= (g_0,p_0)\\
(G_i,P_i) &= (g_i,p_i) \circledast (G_{i-1},P_{i-1})
\end{split}
\end{equation}







\subsection{Παράλληλοι Προθεματικοί Αθροιστές}






\subsection{Δέντρα-Δομές Προθεμάτων}
