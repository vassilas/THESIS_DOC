\thispagestyle{empty}

\renewcommand{\abstractname}{Αbstract}
\begin{abstract}

\end{abstract}


\renewcommand{\abstractname}{Περίληψη}
\begin{abstract}
Η πρόσθεση είναι η βασικότερη αριθμητική πράξη, αποτελώντας τμήμα των περισσότερων 
ψηφιακών συστημάτων. Επίσης, είναι η πιο συχνή πράξη που εκτελείται από μία επεξεργαστική
μονάδα. Η επιτάχυνση της απασχολεί την επιστήμη των ψηφιακών κυκλωμάτων για αρκετά χρόνια.
Το μεγαλύτερο χρονικό κόστος καταλαμβάνεται από τον υπολογισμό των κρατουμένων ανά ζεύγος ψηφίων. 
Την δημοφιλέστερη δομή άθροισης αποτελούν οι αθροιστές πρόβλεψης κρατουμένου (Carry-Look-Ahead CLA).

Με βάση την δομή των CLA αναπτύχθηκαν και οι προθεματική αθροιστές οι οποίοι σύστησαν ένα μεγάλο 
σύνολο από αρχιτεκτονικές δέντρων με σκοπό την παραλληλοποίηση. 
Αν και καταλαμβάνουν μεγαλύτερη ενέργεια και εμβαδόν συγκριτικά με τις απλές δομές, 
έχουν αισθητά μικρότερη καθυστέρηση. Ο H. Ling, τπ 1981, πρότεινε μία παραγοντοποίηση
η οποία είχε εφαρμογή στους προθεματικούς αθροιστές αθροιστές (Ling αθροιστές), 
μειώνοντας την πολυπλοκότητα ενώ, το 2004, οι R. Jackson και S. Talwar γενίκευσαν την 
παραγοντοποίηση αυτή προσφέροντας την δυνατότητα επιπλέον μείωσης της πολυπλοκότητας 
υπολογισμού των κρατουμένων (Jackson αθροιστές).

Ένα μεγάλο τμήμα των αθροιστών αποτελούν οι αθροιστές υπολοίπου. 
Σκοπός της παρούσας εργασίας είναι η επιτάχυνση των αθροιστών υπολοίπου $2^n-1$ συνδυάζοντας
την καλύτερη τοπολογία που έχει προταθεί και την παραγοντοποίηση που προσφέρουν οι αθροιστές Jackson.
Αρχικά γίνεται μια συνοπτική αναφορά σε βασικές αρχιτεκτονικές πρόσθεσης και 
αναλύονται οι αποδοτικότερες χρονικά τοπολογίες με επιπλέον αναφορά στις σχεδιαστικές δομές
των αθροιστών υπολοίπου $2^n-1$. Αναλύονται σε βάθος οι παραγοντοποιήσεις Ling και Jackson 
καθώς η ανάπτυξη των αθροιστών προς σύγκριση βασίζονται σε αυτές.
\end{abstract}