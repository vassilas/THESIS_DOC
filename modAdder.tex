\section{Αθροιστής υπολοίπου $2^n-1$}
Η αριθμητική υπολοίπου αφορά το υπόλοιπο ενός αριθμού X διαιρεμένου με έναν Y, εναλλακτικά
αφαιρείτε από το Χ το Y μέχρι X < Y. Την πράξη αυτή την συμβολίζουμε ως 
\begin{equation*}
    X\ mod\ Y
\end{equation*}
Αριθμητικές υπολοίπου έχουν εφαρμογές σε ένα μεγάλο πλήθος εφαρμογών εφόσον αποτελούν και 
την βάση για τα Residue Number Systems (RNS). Επίσης αποτελούν μέρος της ψηφιακής
επεξεργασίας σημάτων και ψηφιακών φίλτρων , κρυπτογραφίας , σε τεχνικές ανίχνευσης και διόρθωσης σφάλματος καθώς και σε υψηλών ταχυτήτων δίκτυα. Η διαδική άθροιση είναι σε 
αριθμητική υπολοίπου και γράφεται $(A+B)\ mod\ n$ όπου n είναι το πλήθος των δυαδικών 
ψηφίων των A και B. 
Στην παρούσα ενότητα θα μελετηθούν οι αθροιστές υπολοίπου $2^n-1$ όπου η επιτάχυνση τους είναι ο σκοπός μας.


\subsection{Basic Operation}
Ο μαθηματικός υπολογισμός του αθροίσματος υπολοίπου $2^n-1$ στην πραγματικότητα είναι 
ένας υπο-συνθήκη υπολογισμός με συνθήκη $A+B < 2^n$ και ορίζεται ως 
\begin{equation}
(A+B)\, mod\, (2^n-1) = 
\begin{cases}
    (A+B)\, mod\; 2^n       , &  A+B < 2^n\\
    (A+B)\, mod\; 2^n + 1   , & A+B \geq 2^n
\end{cases}
\end{equation}
\\\\
\textcolor{red}{[Γιατί είναι αυτός ο ορισμός ? \\ Γράψε την εξήγηση]}
\\\\
Υπάρχουν διάφοροι τρόποι για να υπολογιστεί στο υλικό το αποτέλεσμα 
ενός αθροιστή υπολοίπου $2^n-1$.

Η πιο απλή ιδέα είναι αποτελείται από δύο αθροιστές όπου ο πρώτος δεν έχει
κρατούμενο εισόδου , παίρνει ως είσοδο τα Α και Β και η έξοδος του τροφοδοτεί
την είσοδο του δεύτερου αθροιστή με δεύτερο όρισμα τον μηδενικό αριθμό
και κρατούμενο εισόδου το κρατούμενο εξόδου του πρώτου αθροιστή. Το άθροισμα 
του δεύτερου αθροιστή είναι και το ζητούμενο. Στο παρακάτω σχηματικό αποτυπώνεται
αυτή η απλή αρχιτεκτονική που περιγράφηκε.
\\\\
\textcolor{red}{[Βάλε εικόνα του απλού αθροιστή $2^n-1$]}
\\\\
Η παραπάνω τεχνική έχει πολύ μεγάλη χρονική καθυστέρηση διότι υπάρχουν δύο 
επίπεδα αθροιστών. Για να μειωθεί ο χρόνος που απαιτείται για να οδηγηθεί η έξοδος
με το σωστό αποτέλεσμα μπορούμε να εκτελέσουμε παράλληλα δυο προσθέσεις του Α και Β
με τον ένα αθροιστή να έχει κρατούμενο εισόδου και τον άλλο να μην έχει. Τα αποτελέσματα 
των δύο αθροιστών θα οδηγούνται σε έναν πολυπλέκτη με είσοδο επιλογής το κρατούμενο 
εισόδου του αθροιστή χωρίς κρατούμενο εισόδου. Αν η είσοδος επιλογής είναι ενεργή 
τότε θα επιλέγεται η έξοδος του αθροιστή με κρατούμενο εισόδου όπως φαίνεται στην 
παρακάτω εικόνα.
\\\\
\textcolor{red}{[Βάλε εικόνα του Επιλογής κρατουμένου αθροιστή $2^n-1$]}
\\\\

\subsection{Architectures Improvements}
















