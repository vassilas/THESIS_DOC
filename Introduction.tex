
\section{Εισαγωγη}

Οι αριθμητικές μονάδες είναι βασικά στοιχεία στο σύνολο του υλικού των ηλεκτρονικών υπολογιστών . Οι περισσότερες επεξεργαστικές μονάδες έχουν ως βασικό δομικό στοιχείο την Αριθμητική Λογική Μονάδα (ΑΛΜ) ή Arithmetic Logic Unit (ALU) η οποία είναι υπεύθυνη για την υλοποίηση βασικών αριθμητικών πράξεων όπως πρόσθεση και πολλαπλασιασμό αλλά και λογικών πράξεων όπως OR, AND και XOR . Αυτές οι μονάδες είναι κρίσιμες στην ορθή λειτουργία ενός συστήματος όχι μόνο ως προς την λειτουργικότητα αλλά και ως προς την ταχύτητα , κατανάλωση ενέργειας και χώρου. Στην παρούσα διπλωματική εργασία θα αναλυθούν οι αθροιστές και συγκεκριμένα μια υποομάδα αυτών , οι 2n-1 αθροιστές . Η ομάδα αυτών των αθροιστών είναι χρήσιμη στα Συστήματα Αριθμού Υπολειμμάτων ή Residue Number Systems (RNS) , Συστήματα Ανθεκτικά σε Σφάλματα ή Fault-Tolerant Systems , στην ανίχνευση σφάλματος στα συστήματα δικτύου καθώς και στις αριθμητικές πράξεις κινητής υποδιαστολής \cite{j32_bit} \cite{j64_bit} \cite{modulo}.