\section{Μετρήσεις και Αποτελέσματα}

Σε αυτό το κεφάλαιο παρουσιάζονται οι μετρήσεις χρόνου, χώρου και κατανάλωσης των αθροιστών 
που περιγράφηκαν στο προηγούμενο εδάφιο. Για τις μετρήσεις αυτές χρησιμοποιήθηκε το εργαλείο
σύνθεσης Design Compiler, σε τοπογραφική και μη ρύθμιση.







\subsection{Εργαλείο Μετρήσεων}

% \subsubsection*{Logic Synthesis}
Η λογική σύνθεση είναι μια υπό-διαδικασία του ηλεκτρονικού αυτόματου σχεδιασμού,
κατά την οποία η αφηρημένη περιγραφή του ψηφιακού συστήματος 
μετατρέπεται σε κύκλωμα λογικών πυλών. Λογισμικά υπεύθυνα για αυτή την διαδικασία
ονομάζονται εργαλεία σύνθεσις ( Synthesis tools ) και είτε παράγουν bitstreams για FPGAs
είτε τα απαραίτητα αρχεία για την δημιουργία ASICs. Σε κάθε περίπτωση παράγουν
επίσης κάποιες εκτιμήσεις σχετικά με την ταχύτητα, αξιοποίηση πόρων, κατανάλωση ενέργειας
και άλλα.

Ένα από τα δημοφιλέστερα εργαλεία για σύνθεση αποτελεί το Design Compiler της Synopsys,
το οποίο θα χρησιμοποιηθεί και για τις μετρήσεις των αθροιστών που αναπτύχθηκαν σε
αυτή την διαδικασία. Το Design Compiler ουσιαστικα μεταφράζει τον HDL κώδικα σε λογικές
μονάδες, υλοποιώντας μια διαδικασία αντιστοίχησης από των κώδικα στο φυσικό επίπεδο.

Οι λογικές μονάδες, πάνω στις οποίες θα αποτυπωθεί το κύκλωμα που σχεδιάστηκε, 
αποτελούνται από μικρά και απλά λογικά κυκλώματα, τα οποία μπορεί να περιέχουν 
μια λογική πύλη, ή να υλοποιούν μια απλή λογική συνάρτηση, όπως για παράδειγμα 
το λογικό στοιχείο AO21X1 το οποίο περιγράφεται απο την λογική συνάρτηση
\begin{equation*}
    Q = ( IN1 * IN2 ) + IN3
\end{equation*}
Αυτά τα στοιχεία έχουν υλοποιηθεί σε φυσικό επίπεδο και έχει μετρηθεί αναλυτικά η συμπεριφορά
τους. Μία τεχνολογική βιβλιοθήκη είναι ένα σύνολο από λογικά στοιχεία στην μορφή που περιγράφηκε
παραπάνω. 

Για την σύνθεση των αθροιστών έγινε χρήση της βιβλιοθήκης "Digital Standard Cell Library" 
των 32nm της Synopsys. Στην βιβλιοθήκη αυτή υλοποιούνται βασικές πύλες όπως AND, NAND, 
OR, NOR, XOR, XNOR, καθώς και άπλες λογικές συναρτήσεις όπως AND-OR, OR-AND, πολυπλέκτες 
δυο σε ένα, αποκωδικοποιητές, ήμι-αθροιστές και πλήρη αθροιστές και πολλά άλλα στοιχεία.
Για κάθε ένα λογικό κελί που περιέχει η βιβλιοθήκη δίνεται μια αναλυτική αναφορά
για τους χρόνους καθυστέρησης και την κατανάλωση ενέργειας ανάλογα με την συμπεριφορά της εισόδου,
το εμβαδόν που καταλαμβάνει το λογικό κύκλωμά, καθώς και άλλες πληροφορίες που αφορούν την 
συχνότητα λειτουργίας, θερμοκρασία, τάση και χωρητικότητα.









\subsection{Τακτική Μεταγλώττισης}

Για λόγους βελτιστοποίησης απαιτείται η κατάλληλη επιλογή της τακτικής μεταγλώττισης (compile strategy) από το εργαλείο σύνθεσης. Το Design Compiler προσφέρει τις παρακάτω επιλογές :
\begin{itemize}
  \item Top-Down Compilation\\
  Ο υψηλότερος ιεραρχικά σχεδιασμός μεταγλωττίζεται μαζί με τους επιμέρους από τους οποίους αποτελείται.
  \item Bottom-Up Compilation\\
  Κάθε sub-module μεταγλωττίζεται ξεκινώντας το κατώτερο ιεραρχικά επίπεδο και συνεχίζοντας στα επόμενα επίπεδα μέχρι να μεταγλωττιστεί και ο υψηλότερος ιεραρχικά σχεδιασμός.
  \item Mixed Compilation\\
  Συνδυασμός των παραπάνω δύο.
\end{itemize}
Για τους σκοπούς της παρούσας εργασίας χρησιμοποιείται η τεχνική Bottom-up, αποσκοπώντας στον περιορισμό των βελτιστοποιήσεων που παρέχει το εργαλείο μόνο στο εσωτερικό των κόμβων, που είναι και το χαμηλότερο επίπεδο της ιεραρχίας. Πέρα από τους κόμβους δεν επιτρέπεται καμία άλλη συναρτησιακή βελτιστοποίηση. Με αυτήν την στρατηγική το εργαλείο δεν έχει την δυνατότητα τροποποίησης της αρχιτεκτονικής των αθροιστών που αναπτύσσονται.

Επιπλέον, το εργαλείο της Synopsys προσφέρει την δυνατότητα τοπογραφικής μεταγλώττισης και μη. Στην περίπτωση μη-τοπογραφικής μεταγλώττισης αξιοποιούνται οι προδιαγραφές των κυκλωμάτων που παρέχονται από την βιβλιοθήκη και συνυπολογίζονται και επιπλέον λογικές που προστίθενται για την σωστή οδήγηση των πυλών. Στην περίπτωση της τοπογραφικής μεταγλώττισης συνυπολογίζονται στα μοντέλα σύνθεσης οι καθυστερήσεις και ο χώρος των καλωδίων μεταξύ των έτοιμων σχεδιασμένων κυκλωμάτων που παρέχονται από την βιβλιοθήκη, καθώς και η τοποθέτηση των κόμβων.











\clearpage
\subsection{Μετρήσεις}
Στην παράγραφο αυτή καταγράφονται τα αποτελέσματα των, τοπογραφικών και μη, μετρήσεων των αθροιστών σε μορφή πινάκων. Σημειώνεται πως για τις μετρήσεις δεν δόθηκε κανένας περιορισμός στο εργαλείο σύνθεσης (Παράρτημα \ref{appendix:scripts}). Οι μονάδες μέτρησης καθυστέρησης, εμβαδού και κατανάλωση ενέργειας είναι $ns$, $nm^2$ και $pw$ αντίστοιχα.
\begin{multicols}{2}
[\subsubsection{Μη-Τοπογραφικές Μετρήσεις}]
%------------------------------------------------------------------------------
\begin{table}[H]
\centering
     \begin{tabular}{||p{1.2cm} | p{0.7cm}  p{1cm}  p{1cm} ||} 
        \hline
         & Delay & Area & Power \\ [0.5ex] 
        \hline\hline
        Prefix  & 0.39  & 216.70    & 50.25 \\ 
        \hline
        Ling    & 0.39  & 237.66    & 49.91 \\
        \hline
        Jackson & 0.35  & 281.51    & 58.69 \\
        \hline
    \end{tabular}
\caption{Μετρήσεις 8-bit}
\label{result_table_8}
\end{table}
%------------------------------------------------------------------------------
\begin{table}[H]
\centering
     \begin{tabular}{||p{1.2cm} | p{0.7cm} p{1cm} p{1cm} ||} 
        \hline
        & Delay & Area & Power \\ [0.5ex] 
        \hline\hline
        Prefix  & 0.50  & 573.10    & 116.68 \\ 
        \hline
        Ling    & 0.48  & 590.69    & 119.84 \\
        \hline
        Jackson & 0.43  & 634.68    & 129.00 \\
        \hline
    \end{tabular}
\caption{Μετρήσεις 16-bit}
\label{result_table_16}
\end{table}
%------------------------------------------------------------------------------
\begin{table}[H]
\centering
     \begin{tabular}{||p{1.2cm} | p{0.7cm} p{1cm} p{1cm} ||} 
        \hline
        & Delay & Area & Power \\ [0.5ex] 
        \hline\hline
        Prefix  & 0.55  & 1319.29    & 265.71 \\ 
        \hline
        Ling    & 0.58  & 1354.47    & 275.13 \\
        \hline
        Jackson & 0.48  & 1679.46    & 336.28 \\
        \hline
    \end{tabular}
\caption{Μετρήσεις 32-bit}
\label{result_table_32}
\end{table}
%------------------------------------------------------------------------------
\begin{table}[H]
\centering
     \begin{tabular}{||p{1.2cm} | p{0.7cm} p{1cm} p{1cm} ||} 
        \hline
         & Delay & Area & Power \\ [0.5ex] 
        \hline\hline
        Prefix  & 0.67  & 3100.09    & 596.56 \\ 
        \hline
        Ling    & 0.65  & 3170.45    & 612.25 \\
        \hline
        Jackson & 0.57  & 4049.84    & 766.75 \\
        \hline
    \end{tabular}
\caption{Μετρήσεις 64-bit}
\label{result_table_64}
\end{table}
%------------------------------------------------------------------------------
\end{multicols}












\begin{multicols}{2}
[\subsubsection{Μη-Τοπογραφικές Μετρήσεις sparse-2}]

\begin{table}[H]
\centering
     \begin{tabular}{||p{1.2cm} | p{0.7cm} p{1cm} p{1cm}||} 
        \hline
        & Delay & Area & Power \\ [0.5ex] 
        \hline\hline
        Prefix  & 0.44  & 173.73    & 40.2 \\ 
        \hline
        Ling    & 0.47  & 186.01    & 43.6 \\
        \hline
        Jackson & 0.42  & 206.22    & 48.2 \\
        \hline
    \end{tabular}
\caption{Μη-Τοπογραφικές sparse-2 Μετρήσεις 8-bit}
\label{notopo_sparse2_result_table_8}
\end{table}
%------------------------------------------------------------------------------
\begin{table}[H]
\centering
     \begin{tabular}{||p{1.2cm} | p{0.7cm} p{1cm} p{1cm}||} 
        \hline
        & Delay & Area & Power \\ [0.5ex] 
        \hline\hline
        Prefix  & 0.54  & 421.26    & 87.90 \\ 
        \hline
        Ling    & 0.53  & 446.91    & 94.20 \\
        \hline
        Jackson & 0.48  & 496.23    & 103.34 \\
        \hline
    \end{tabular}
\caption{Μη-Τοπογραφικές sparse-2 Μετρήσεις 16-bit}
\label{topo_sparse2_result_table_16}
\end{table}
%------------------------------------------------------------------------------
\begin{table}[H]
\centering
     \begin{tabular}{||p{1.2cm} | p{0.7cm} p{1cm} p{1cm}||} 
        \hline
         & Delay & Area & Power \\ [0.5ex] 
        \hline\hline
        Prefix  & 0.60  & 932.93    & 191.75 \\ 
        \hline
        Ling    & 0.64  & 983.48    &  205.91 \\
        \hline
        Jackson & 0.53  & 1143.91    & 236.01 \\
        \hline
    \end{tabular}
\caption{Μη-Τοπογραφικές sparse-2 Μετρήσεις 32-bit}
\label{topo_sparse2_result_table_32}
\end{table}
%------------------------------------------------------------------------------
\begin{table}[H]
\centering
     \begin{tabular}{||p{1.2cm} | p{0.7cm} p{1cm} p{1cm}||} 
        \hline
         & Delay & Area & Power \\ [0.5ex] 
        \hline\hline
        Prefix  & 0.72  & 2088.89    & 416.18 \\ 
        \hline
        Ling    & 0.71  & 2191.47    & 442.85 \\
        \hline
        Jackson & 0.61  & 2619.22    & 518.93 \\
        \hline
    \end{tabular}
\caption{Μη-Τοπογραφικές sparse-2 Μετρήσεις 64-bit}
\label{topo_sparse2_result_table_64}
\end{table}
%------------------------------------------------------------------------------
\end{multicols}




\clearpage







\begin{multicols}{2}
[\subsubsection{Μη-Τοπογραφικές Μετρήσεις sparse-4}]
%------------------------------------------------------------------------------
\begin{table}[H]
\centering
     \begin{tabular}{||p{1.2cm} | p{0.7cm}  p{1cm}  p{1cm} ||} 
        \hline
         & Delay & Area & Power \\ [0.5ex] 
        \hline\hline
        Prefix  & 0.57  & 162.32    & 37.43 \\ 
        \hline
        Ling    & 0.47  & 191.83    & 43.01 \\
        \hline
        Jackson & 0.40  & 228.68    & 47.46 \\
        \hline
    \end{tabular}
\caption{Μη-Τοπογραφικές sparse-4 Μετρήσεις 8-bit}
\label{topo_sparse4_result_table_8}
\end{table}
%------------------------------------------------------------------------------
\begin{table}[H]
\centering
     \begin{tabular}{||p{1.2cm} | p{0.7cm} p{1cm} p{1cm} ||} 
        \hline
        & Delay & Area & Power \\ [0.5ex] 
        \hline\hline
        Prefix  & 0.67  & 346.41    & 73.69 \\ 
        \hline
        Ling    & 0.52  & 411.05    & 84.11 \\
        \hline
        Jackson & 0.48  & 434.99    & 88.50 \\
        \hline
    \end{tabular}
\caption{Μη-Τοπογραφικές sparse-4 Μετρήσεις 16-bit}
\label{topo_sparse4_result_table_16}
\end{table}
%------------------------------------------------------------------------------
\begin{table}[H]
\centering
     \begin{tabular}{||p{1.2cm} | p{0.7cm} p{1cm} p{1cm} ||} 
        \hline
        & Delay & Area & Power \\ [0.5ex] 
        \hline\hline
        Prefix  & 0.73  & 779.54    & 163.35 \\ 
        \hline
        Ling    & 0.63  & 902.82    & 184.92 \\
        \hline
        Jackson & 0.52  & 983.48    & 198.59 \\
        \hline
    \end{tabular}
\caption{Μη-Τοπογραφικές sparse-4 Μετρήσεις 32-bit}
\label{topo_sparse4_result_table_32}
\end{table}
%------------------------------------------------------------------------------
\begin{table}[H]
\centering
     \begin{tabular}{||p{1.2cm} | p{0.7cm} p{1cm} p{1cm} ||} 
        \hline
         & Delay & Area & Power \\ [0.5ex] 
        \hline\hline
        Prefix  & 0.85  & 1587.56    & 326.94 \\ 
        \hline
        Ling    & 0.70  & 1846.12    & 369.31 \\
        \hline
        Jackson & 0.62  & 2202.05    & 426.12 \\
        \hline
    \end{tabular}
\caption{Μη-Τοπογραφικές sparse-4 Μετρήσεις 64-bit}
\label{topo_sparse4_result_table_64}
\end{table}
%------------------------------------------------------------------------------
\end{multicols}















% \begin{multicols}{2}
% [\subsubsection{Τοπογραφικές Μετρήσεις}]

% \begin{table}[H]
% \centering
%      \begin{tabular}{||p{1.2cm} | p{0.7cm} p{1cm} p{1cm} ||} 
%         \hline
%          & Delay & Area & Power \\ [0.5ex]
%         \hline\hline
%         Prefix  & 0.393  & 150.7    & 38 \\ 
%         \hline
%         Ling    & 0.433  & 152.5    & 39 \\
%         \hline
%         Jackson & 0.427  & 156      & 37 \\
%         \hline
%     \end{tabular}
% \caption{Τοπογραφικές Μετρήσεις 8-bit}
% \label{topo_result_table_8}
% \end{table}
% %------------------------------------------------------------------------------
% \begin{table}[H]
% \centering
%      \begin{tabular}{||p{1.2cm} | p{0.7cm} p{1cm} p{1cm} ||} 
%         \hline
%          & Delay & Area & Power \\ [0.5ex] 
%         \hline\hline
%         Prefix  & 0.589  & 384    & 97 \\ 
%         \hline
%         Ling    & 0.597  & 397    & 103 \\
%         \hline
%         Jackson & 0.542  & 449    & 120 \\
%         \hline
%     \end{tabular}
% \caption{Τοπογραφικές Μετρήσεις 16-bit}
% \label{topo_result_table_16}
% \end{table}
% %------------------------------------------------------------------------------
% \begin{table}[H]
% \centering
%      \begin{tabular}{||p{1.2cm} | p{0.7cm} p{1cm} p{1cm} ||} 
%         \hline
%          & Delay & Area & Power \\ [0.5ex]
%         \hline\hline
%         Prefix  & 0.667  & 949.8    & 252.9 \\ 
%         \hline
%         Ling    & 0.747  & 917.2    & 245 \\
%         \hline
%         Jackson & 0.652  & 1109    & 299.6 \\
%         \hline
%     \end{tabular}
% \caption{Τοπογραφικές Μετρήσεις 32-bit}
% \label{topo_result_table_32}
% \end{table}
% %------------------------------------------------------------------------------
% \begin{table}[H]
% \centering
%      \begin{tabular}{||p{1.2cm} | p{0.7cm} p{1cm} p{1cm} ||} 
%         \hline
%          & Delay & Area & Power \\ [0.5ex] 
%         \hline\hline
%         Prefix  & 0.726  & 2725    & 736 \\ 
%         \hline
%         Ling    & 0.824  & 2349    & 625 \\
%         \hline
%         Jackson & 0.863  & 2161    & 550 \\
%         \hline
%     \end{tabular}
% \caption{Τοπογραφικές Μετρήσεις 64-bit}
% \label{topo_result_table_64}
% \end{table}
% %------------------------------------------------------------------------------
% \end{multicols}

\begin{multicols}{2}
[\subsubsection{Τοπογραφικές Μετρήσεις}]

\begin{table}[H]
\centering
     \begin{tabular}{||p{1.2cm} | p{0.7cm} p{1cm} p{1cm} ||} 
        \hline
         & Delay & Area & Power \\ [0.5ex]
        \hline\hline
        Prefix & 0.289 & 166 & 43.56 \\
        \hline
        Ling & 0.313 & 178 & 48.60 \\
        \hline
        Jackson & 0.335 & 215 & 61.85 \\
        \hline
    \end{tabular}
\caption{Τοπογραφικές Μετρήσεις 8-bit}
\label{topo_result_table_8}
\end{table}
%------------------------------------------------------------------------------
\begin{table}[H]
\centering
     \begin{tabular}{||p{1.2cm} | p{0.7cm} p{1cm} p{1cm} ||} 
        \hline
         & Delay & Area & Power \\ [0.5ex] 
        \hline\hline
        Prefix & 0.384 & 398 & 102.85 \\
        \hline
        Ling & 0.404 & 422 & 117.99 \\
        \hline
        Jackson & 0.397 & 455 & 133.82 \\
        \hline
    \end{tabular}
\caption{Τοπογραφικές Μετρήσεις 16-bit}
\label{topo_result_table_16}
\end{table}
%------------------------------------------------------------------------------
\begin{table}[H]
\centering
     \begin{tabular}{||p{1.2cm} | p{0.7cm} p{1cm} p{1cm} ||} 
        \hline
         & Delay & Area & Power \\ [0.5ex]
        \hline\hline
        Prefix & 0.439 & 943 & 239.80 \\
        \hline
        Ling & 0.462 & 992 & 261.15 \\
        \hline
        Jackson & 0.467 & 1268 & 357.11 \\
        \hline
    \end{tabular}
\caption{Τοπογραφικές Μετρήσεις 32-bit}
\label{topo_result_table_32}
\end{table}
%------------------------------------------------------------------------------
\begin{table}[H]
\centering
     \begin{tabular}{||p{1.2cm} | p{0.7cm} p{1cm} p{1cm} ||} 
        \hline
         & Delay & Area & Power \\ [0.5ex] 
        \hline\hline
        Prefix & 0.548 & 2147 & 522.28 \\
        \hline
        Ling & 0.583 & 2244 & 587.19 \\
        \hline
        Jackson & 0.581 & 2960 & 821.75 \\
        \hline
    \end{tabular}
\caption{Τοπογραφικές Μετρήσεις 64-bit}
\label{topo_result_table_64}
\end{table}
%------------------------------------------------------------------------------
\end{multicols}








\clearpage
\begin{multicols}{2}
[\subsubsection{Τοπογραφικές Μετρήσεις sparse-2}]

\begin{table}[H]
\centering
     \begin{tabular}{||p{1.2cm} | p{0.7cm} p{1cm} p{1cm} ||} 
        \hline
         & Delay & Area & Power \\ [0.5ex]
        \hline\hline
        Prefix & 0.330 & 134 & 35.12 \\
        \hline
        Ling & 0.335 & 143 & 38.74 \\
        \hline
        Jackson & 0.343 & 161 & 44.82 \\
        \hline
    \end{tabular}
\caption{Τοπογραφικές sparse-2 Μετρήσεις 8-bit}
\label{topo_sparse2_result_table_8}
\end{table}
%------------------------------------------------------------------------------
\begin{table}[H]
\centering
     \begin{tabular}{||p{1.2cm} | p{0.7cm} p{1cm} p{1cm} ||} 
        \hline
         & Delay & Area & Power \\ [0.5ex] 
        \hline\hline
        Prefix & 0.423 & 300 & 78.92 \\
        \hline
        Ling & 0.421 & 319 & 89.23 \\
        \hline
        Jackson & 0.426 & 355 & 102.14 \\
        \hline
    \end{tabular}
\caption{Τοπογραφικές sparse-2 Μετρήσεις 16-bit}
\label{topo_sparse2_result_table_16}
\end{table}
%------------------------------------------------------------------------------
\begin{table}[H]
\centering
     \begin{tabular}{||p{1.2cm} | p{0.7cm} p{1cm} p{1cm} ||} 
        \hline
         & Delay & Area & Power \\ [0.5ex]
        \hline\hline
        Prefix & 0.476 & 675 & 172.43 \\
        \hline
        Ling & 0.486 & 711 & 189.33 \\
        \hline
        Jackson & 0.465 & 849 & 232.75 \\
        \hline
    \end{tabular}
\caption{Τοπογραφικές sparse-2 Μετρήσεις 32-bit}
\label{topo_sparse2_result_table_32}
\end{table}
%------------------------------------------------------------------------------
\begin{table}[H]
\centering
     \begin{tabular}{||p{1.2cm} | p{0.7cm} p{1cm} p{1cm} ||} 
        \hline
         & Delay & Area & Power \\ [0.5ex] 
        \hline\hline
        Prefix & 0.571 & 1480 & 368.42 \\
        \hline
        Ling & 0.567 & 1553 & 407.71 \\
        \hline
        Jackson & 0.559 & 1911 & 516.89 \\
        \hline
    \end{tabular}
\caption{Τοπογραφικές sparse-2 Μετρήσεις 64-bit}
\label{topo sparse2_result_table_64}
\end{table}
%------------------------------------------------------------------------------
\end{multicols}








\begin{multicols}{2}
[\subsubsection{Τοπογραφικές Μετρήσεις sparse-4}]

\begin{table}[H]
\centering
     \begin{tabular}{||p{1.2cm} | p{0.7cm} p{1cm} p{1cm} ||} 
        \hline
         & Delay & Area & Power \\ [0.5ex]
        \hline\hline
        Prefix & 0.429 & 127 & 33.38 \\
        \hline
        Ling & 0.419 & 130 & 34.49 \\
        \hline
        Jackson & 0.427 & 150 & 42.43 \\
        \hline
    \end{tabular}
\caption{Τοπογραφικές sparse-4 Μετρήσεις 8-bit}
\label{topo_sparse4_result_table_8}
\end{table}
%------------------------------------------------------------------------------
\begin{table}[H]
\centering
     \begin{tabular}{||p{1.2cm} | p{0.7cm} p{1cm} p{1cm} ||} 
        \hline
         & Delay & Area & Power \\ [0.5ex] 
        \hline\hline
        Prefix & 0.499 & 252 & 66.87 \\
        \hline
        Ling & 0.501 & 261 & 71.30 \\
        \hline
        Jackson & 0.505 & 279 & 77.04 \\
        \hline
    \end{tabular}
\caption{Τοπογραφικές sparse-4 Μετρήσεις 16-bit}
\label{topo_sparse4_result_table_16}
\end{table}
%------------------------------------------------------------------------------
\begin{table}[H]
\centering
     \begin{tabular}{||p{1.2cm} | p{0.7cm} p{1cm} p{1cm} ||} 
        \hline
         & Delay & Area & Power \\ [0.5ex]
        \hline\hline
        Prefix & 0.560 & 577 & 151.20 \\
        \hline
        Ling & 0.558 & 591 & 157.13 \\
        \hline
        Jackson & 0.550 & 660 & 180.03 \\
        \hline
    \end{tabular}
\caption{Τοπογραφικές sparse-4 Μετρήσεις 32-bit}
\label{topo_sparse4_result_table_32}
\end{table}
%------------------------------------------------------------------------------
\begin{table}[H]
\centering
     \begin{tabular}{||p{1.2cm} | p{0.7cm} p{1cm} p{1cm} ||} 
        \hline
         & Delay & Area & Power \\ [0.5ex] 
        \hline\hline
        Prefix & 0.649 & 1146 & 297.58 \\
        \hline
        Ling & 0.642 & 1183 & 313.25 \\
        \hline
        Jackson & 0.641 & 1476 & 392.74 \\
        \hline
    \end{tabular}
\caption{Τοπογραφικές sparse-4 Μετρήσεις 64-bit}
\label{topo sparse4_result_table_64}
\end{table}
%------------------------------------------------------------------------------
\end{multicols}




% \begin{multicols}{2}
% [\subsubsection{Τοπογραφικές Μετρήσεις sparse-2}]

% \begin{table}[H]
% \centering
%      \begin{tabular}{||p{1.2cm} | p{0.7cm} p{1cm} p{1cm} ||} 
%         \hline
%          & Delay & Area & Power \\ [0.5ex]
%         \hline\hline
%         Prefix  & 0.393  & 150.7    & 38 \\ 
%         \hline
%         Ling    & 0.433  & 152.5    & 39 \\
%         \hline
%         Jackson & 0.427  & 156      & 37 \\
%         \hline
%     \end{tabular}
% \caption{Τοπογραφικές Μετρήσεις 8-bit}
% \label{topo_result_table_8}
% \end{table}
% %------------------------------------------------------------------------------
% \begin{table}[H]
% \centering
%      \begin{tabular}{||p{1.2cm} | p{0.7cm} p{1cm} p{1cm} ||} 
%         \hline
%          & Delay & Area & Power \\ [0.5ex] 
%         \hline\hline
%         Prefix  & 0.589  & 384    & 97 \\ 
%         \hline
%         Ling    & 0.597  & 397    & 103 \\
%         \hline
%         Jackson & 0.542  & 449    & 120 \\
%         \hline
%     \end{tabular}
% \caption{Τοπογραφικές Μετρήσεις 16-bit}
% \label{topo_result_table_16}
% \end{table}
% %------------------------------------------------------------------------------
% \begin{table}[H]
% \centering
%      \begin{tabular}{||p{1.2cm} | p{0.7cm} p{1cm} p{1cm} ||} 
%         \hline
%          & Delay & Area & Power \\ [0.5ex]
%         \hline\hline
%         Prefix  & 0.667  & 949.8    & 252.9 \\ 
%         \hline
%         Ling    & 0.747  & 917.2    & 245 \\
%         \hline
%         Jackson & 0.652  & 1109    & 299.6 \\
%         \hline
%     \end{tabular}
% \caption{Τοπογραφικές Μετρήσεις 32-bit}
% \label{topo_result_table_32}
% \end{table}
% %------------------------------------------------------------------------------
% \begin{table}[H]
% \centering
%      \begin{tabular}{||p{1.2cm} | p{0.7cm} p{1cm} p{1cm} ||} 
%         \hline
%          & Delay & Area & Power \\ [0.5ex] 
%         \hline\hline
%         Prefix  & 0.726  & 2725    & 736 \\ 
%         \hline
%         Ling    & 0.824  & 2349    & 625 \\
%         \hline
%         Jackson & 0.863  & 2161    & 550 \\
%         \hline
%     \end{tabular}
% \caption{Τοπογραφικές Μετρήσεις 64-bit}
% \label{topo_result_table_64}
% \end{table}
% %------------------------------------------------------------------------------
% \end{multicols}
















% \clearpage
% ---------------------------------------------------------
% \subsection{Γραφικές}
% \begin{multicols}{2}
% % ---------------------------------------------------------

% \begin{center}
% \begin{tikzpicture}[scale=0.7]

% \begin{axis}[
%     title={Γράφημα καθυστέρησης},
%     xlabel={Μήκος εισόδου (bits)},
%     ylabel={Χρονικά καθηστέρηση (ns)},
%     xmin=0, xmax=70,
%     ymin=30, ymax=70,
%     xtick={8,16,32,64},
%     ytick={30,40,50,60,70},
%     legend pos=north west,
%     ymajorgrids=true,
%     grid style=dashed,
% ]

% \addplot[
%     color=red,
%     mark=square,
%     ]
%     coordinates {
%     (8,35)(16,43)(32,48)(64,57)
%     };
%     \legend{Jackson}

% \addplot[
%     color=blue,
%     mark=*,
%     ]
%     coordinates {
%     (8,39)(16,48)(32,58)(64,65)
%     };
%     \legend{Ling}

% \addplot[
%     color=green,
%     mark=triangle,
%     ]
%     coordinates {
%     (8,39)(16,50)(32,55)(64,67)
%     };
%     \legend{Prefix}

%   \legend{Jackson,Ling,Prefix}
   
% \end{axis}
% \end{tikzpicture}
% \end{center}



% % ---------------------------------------------------------

% \begin{center}
% \begin{tikzpicture}[scale=0.7]

% \begin{axis}[
%     title={Γράφημα καθυστέρησης, sparse-2},
%     xlabel={Μήκος εισόδου (bits)},
%     % ylabel={Χρονικά καθηστέρηση (ns)},
%     xmin=0, xmax=70,
%     ymin=40, ymax=80,
%     xtick={8,16,32,64},
%     ytick={40,50,60,70,80},
%     legend pos=north west,
%     ymajorgrids=true,
%     grid style=dashed,
% ]

% \addplot[
%     color=red,
%     mark=square,
%     ]
%     coordinates {
%     (8,42)(16,48)(32,53)(64,61)
%     };
%     \legend{Jackson}

% \addplot[
%     color=blue,
%     mark=*,
%     ]
%     coordinates {
%     (8,47)(16,53)(32,64)(64,71)
%     };
%     \legend{Ling}

% \addplot[
%     color=green,
%     mark=triangle,
%     ]
%     coordinates {
%     (8,44)(16,54)(32,60)(64,72)
%     };
%     \legend{Prefix}

%   \legend{Jackson,Ling,Prefix}
   
% \end{axis}
% \end{tikzpicture}
% \end{center}

% \end{multicols}








% \begin{multicols}{2}
% \begin{center}
% \begin{tikzpicture}[scale=0.7]

% \begin{axis}[
%     title={Γράφημα καθυστέρησης, sparse-4},
%     xlabel={Μήκος εισόδου (bits)},
%     ylabel={Χρονικά καθηστέρηση (ns)},
%     xmin=0, xmax=70,
%     ymin=40, ymax=90,
%     xtick={8,16,32,64},
%     ytick={40,50,60,70,80,90},
%     legend pos=north west,
%     ymajorgrids=true,
%     grid style=dashed,
% ]

% \addplot[
%     color=red,
%     mark=square,
%     ]
%     coordinates {
%     (8,40)(16,48)(32,52)(64,62)
%     };
%     \legend{Jackson}

% \addplot[
%     color=blue,
%     mark=*,
%     ]
%     coordinates {
%     (8,47)(16,52)(32,63)(64,70)
%     };
%     \legend{Ling}

% \addplot[
%     color=green,
%     mark=triangle,
%     ]
%     coordinates {
%     (8,57)(16,67)(32,73)(64,85)
%     };
%     \legend{Prefix}

%   \legend{Jackson,Ling,Prefix}
   
% \end{axis}
% \end{tikzpicture}
% \end{center}
% \end{multicols}



% \textcolor{red}{[Τις μονάδες μέτρησής πρέπει να διορθώσεις]}










\clearpage
\subsection{Ανάλυση αποτελεσμάτων}
Τα αποτελέσματα των μετρήσεων, όπως ήταν και αναμενόμενο, διαφέρουν ανάλογα με τον τρόπο μεταγλώττισης (τοπογραφικό και μη). Στην συνέχεια σχολιάζονται και οι δύο περιπτώσεις όσο αφορά την απόδοση. 

\subsubsection{Σχολιασμός μη-τοπογραφικών αποτελεσμάτων}
Σύμφωνα με τα αποτελέσματα των μη-τοπογραφικών μετρήσεων, παρατηρείται θετική απόδοση της ταχύτητας σε όλες τις περιπτώσεις. Η αναμενόμενη κατάταξή, όσο αφορά την ταχύτητα, ξεκινά με μεγαλύτερη καθυστέρηση στις απλές τοπολογίες προθεματικών αθροιστών υπολοίπου, και καταλήγει στις τοπολογίες κατά Jackson. Σε παρελθοντικές έρευνες \cite{1377160} και \cite{4633502} έχει αποδειχθεί πως οι αθροιστές υπολοίπου $2^n-1$ κατά ling είναι πιο αποδοτικοί από τους simple-prefix. Στα αποτελέσματα, όπως φαίνεται και στο γράφημα της εικόνας \ref{graph:non-topo}, υπάρχουν περιπτώσεις όπου οι Ling αθροιστές είναι στην χειρότερη περίπτωση.

\begin{figure}[H]
\begin{center}
\begin{tikzpicture}[scale=0.9]

\begin{axis}[
    title={Γράφημα καθυστέρησης},
    xlabel={Μήκος εισόδου (bits)},
    ylabel={Χρονικά καθηστέρηση (ns)},
    xmin=0, xmax=70,
    ymin=30, ymax=70,
    xtick={8,16,32,64},
    ytick={30,40,50,60,70},
    legend pos=north west,
    ymajorgrids=true,
    grid style=dashed,
]

\addplot[
    color=red,
    mark=square,
    ]
    coordinates {
    (8,35)(16,43)(32,48)(64,57)
    };
    \legend{Jackson}

\addplot[
    color=blue,
    mark=*,
    ]
    coordinates {
    (8,39)(16,48)(32,58)(64,65)
    };
    \legend{Ling}

\addplot[
    color=green,
    mark=triangle,
    ]
    coordinates {
    (8,39)(16,50)(32,55)(64,67)
    };
    \legend{Prefix}

   \legend{Jackson,Ling,Prefix}
% 
\end{axis}


\end{tikzpicture}
\end{center}
\caption{Γράφημα καθυστέρησης - μήκος εισόδου αθροιστών υπολοίπου}   
\label{graph:non-topo}
\end{figure}

Αυτό δεν σημαίνει πως οι μετρήσεις είναι άκυρες. Ο βασικός λόγος που προκαλεί την διαφορά αυτή είναι η επιλογή της βιβλιοθήκης. Οι μετρήσεις της παρούσας εργασίας πραγματοποιήθηκαν σε τεχνολογία 32nm, ενώ οι μετρήσεις που παρουσιάζονται σε προηγούμενες δημοσιεύσεις συγκρίνοντας τους προθεματικούς αθροιστές με αθροιστές Ling βασίζονται σε τεχνολογίες μεγαλύτερες των 100nm. 

Όσο αφορά την επιτάχυνση των αθροιστών, όπου είναι και ο στόχος της παρούσας εργασίας, είναι εμφανές και από το διάγραμμα αλλά και από τους πίνακες των αποτελεσμάτων πως υπάρχει αισθητό προβάδισμα. Πρέπει όμως να σημειωθεί πως όσο αφορά την απαίτηση πόρων και ενέργειας τα αποτελέσματα είναι αρνητικά, κάτι που είναι επίσης αναμενόμενο εφόσον για την υλοποίηση της παραγοντοποίησης Jackson είναι αναγκαίο να εισαχθούν στο σχεδιασμό επιπλέον κόμβοι για τον υπολογισμό των σημάτων D.

Στις παραπάνω τρεις αρχιτεκτονικές προς σύγκριση, εφαρμόστηκε η ίδια ακριβός τεχνική αραίωσης, sparse-2 και sparse-4, με σκοπό την μείωση του εμβαδού. Στα γραφήματα \ref{graph:non-topo-sparse-2} και \ref{graph:non-topo-sparse-4} παρουσιάζονται τα διαγράμματα καθυστέρησης - μήκος εισόδου για sparse-2 και sparse-4 αντίστοιχα.

\begin{figure}[H]
\begin{center}
\begin{tikzpicture}[scale=0.9]

\begin{axis}[
    title={Γράφημα καθυστέρησης, sparse-2},
    xlabel={Μήκος εισόδου (bits)},
    % ylabel={Χρονικά καθηστέρηση (ns)},
    xmin=0, xmax=70,
    ymin=40, ymax=80,
    xtick={8,16,32,64},
    ytick={40,50,60,70,80},
    legend pos=north west,
    ymajorgrids=true,
    grid style=dashed,
]

\addplot[
    color=red,
    mark=square,
    ]
    coordinates {
    (8,42)(16,48)(32,53)(64,61)
    };
    \legend{Jackson}

\addplot[
    color=blue,
    mark=*,
    ]
    coordinates {
    (8,47)(16,53)(32,64)(64,71)
    };
    \legend{Ling}

\addplot[
    color=green,
    mark=triangle,
    ]
    coordinates {
    (8,44)(16,54)(32,60)(64,72)
    };
    \legend{Prefix}

   \legend{Jackson,Ling,Prefix}
   
\end{axis}
\end{tikzpicture}
\end{center}
\caption{Γράφημα καθυστέρησης - μήκος εισόδου αθροιστών υπολοίπου sparse-2}   
\label{graph:non-topo-sparse-2}
\end{figure}

Η κατάταξη ταχύτητας δεν επηρεάζεται, αλλά κάθε αθροιστής επιβαρύνεται με ένα χρονικό διάστημα κερδίζοντας σε εμβαδόν. Αξιοσημείωτο είναι το αποτέλεσμα στην περίπτωση sparse-4, όπου οι αθροιστές παρουσιάζουν μεγαλύτερες διαφορές, όσο αφορά τον χρόνο, σε σχέση με τις δύο προηγούμενες περιπτώσεις, ενώ το εμβαδόν τους δεν έχει την αντίστοιχη απόκλιση.

\begin{figure}[H]


\begin{center}
\begin{tikzpicture}[scale=0.9]

\begin{axis}[
    title={Γράφημα καθυστέρησης, sparse-4},
    xlabel={Μήκος εισόδου (bits)},
    ylabel={Χρονικά καθηστέρηση (ns)},
    xmin=0, xmax=70,
    ymin=40, ymax=90,
    xtick={8,16,32,64},
    ytick={40,50,60,70,80,90},
    legend pos=north west,
    ymajorgrids=true,
    grid style=dashed,
]

\addplot[
    color=red,
    mark=square,
    ]
    coordinates {
    (8,40)(16,48)(32,52)(64,62)
    };
    \legend{Jackson}

\addplot[
    color=blue,
    mark=*,
    ]
    coordinates {
    (8,47)(16,52)(32,63)(64,70)
    };
    \legend{Ling}

\addplot[
    color=green,
    mark=triangle,
    ]
    coordinates {
    (8,57)(16,67)(32,73)(64,85)
    };
    \legend{Prefix}

   \legend{Jackson,Ling,Prefix}
   
\end{axis}
\end{tikzpicture}
\end{center}
\caption{Γράφημα καθυστέρησης - μήκος εισόδου αθροιστών υπολοίπου sparse-4}   
\label{graph:non-topo-sparse-4}
\end{figure}





\subsubsection{Σχολιασμός τοπογραφικών αποτελεσμάτων}
Στην περίπτωση των τοπογραφικών μετρήσεων τα αποτελέσματα είναι αρνητικά κατά μέσο όρο με ελάχιστες εξαιρέσεις. Συγκεκριμένα στις sparse-2 και sparse-4 τεχνικές αραίωσης, στους αθροιστές των 32 και 64 δυαδικών ψηφίων παρατηρείται θετική απόδοση όσο αφορά την καθυστέρηση. Στις υπόλοιπες περιπτώσεις τα αποτελέσματα είναι αρνητικά και σε μερικές μη αναμενόμενα, διακρίνοντας ανάποδη χρονική κατάταξη. 

Η προφανής εξήγηση των ανομοιόμορφων αποτελεσμάτων μεταξύ τοπογραφικών και μη μετρήσεων οφείλεται στην τεχνολογία που χρησιμοποιήθηκε. Σύμφωνα με έρευνες που πραγματοποιήθηκαν στο παρελθόν \cite{electronics6040078}, καθώς και στην δημοσίευση του Saraswat και Mohammadi \cite{1051729}, διατυπώνεται πως στις προχωρημένες τεχνολογίες η καθυστέρηση λόγω των διασυνδέσεων έχει γίνει αρκετά σημαντικότερη από τις καθυστερήσεις των λογικών πυλών. Στην εικόνα \ref{graph:gap_between_interconection_and_gates_delay} δίνεται ένα γράφημα που δείχνει την διαφορά αυτή χρησιμοποιώντας τα στοιχεία από τις έρευνες που αναφέρθηκαν.

\begin{figure}[H]
\begin{center}
\begin{tikzpicture}[scale=1]

\begin{axis}[
    width=9cm,
    title={Γράφημα καθυστέρησης πυλών και διασυνδέσεων},
    xlabel={Τεχνολογία (nm)},
    ylabel={Χρονικά καθηστέρηση (ps)},
    ymin=0, ymax=45,
    xtick={650,500,350,250,180,100},
    ytick={0,5,10,15,20,25,30,35,40,45},
    legend pos=north east,
    ymajorgrids=true,
    grid style=dashed,
]

\addplot[
    color=black,
    mark=square,
    ]
    coordinates {
    (650,17)(500,14)(350,9)(250,7)(180,5)(130,4)(100,3)
    };
    \legend{Gate}



\addplot[
    color=black,
    mark=triangle,
    ]
    coordinates {
    (650,2)(500,3)(350,5)(250,7)(180,13)(130,23)(100,39)
    };
    \legend{Interconection}

   \legend{Gate,Interconection}
% 
\end{axis}


\end{tikzpicture}
\end{center}
\caption{Γράφημα καθυστέρησης - μήκος εισόδου αθροιστών υπολοίπου}   
\label{graph:gap_between_interconection_and_gates_delay}
\end{figure}


Στους αθροιστές Jackson σημειώνεται αρκετά μεγαλύτερο πλήθος κόμβων σε σχέση με τους υπόλοιπους δύο σχεδιασμούς. Συνεπάγεται, λοιπόν, πως και το πλήθος των διασυνδέσεων θα είναι μεγαλύτερο καθώς και λόγο δυσκολίας δρομολόγησης ίσως και πιο περίπλοκο.




\subsection{Συμπεράσματα}

Η χρονική απόδοση υπολογισμού του αθροίσματος υπολοίπου $2^n-1$ στους αθροιστές, που αναπτύχθηκαν στην παρούσα εργασία, είναι θετική στις περισσότερες περιπτώσει. Εφόσον στις μη-τοπογραφικές μετρήσεις το αποτελέσματα είναι πάντα θετικά, η τελικές προδιαγραφές των Jackson $2^n-1$ που αναπτύχθηκαν εξαρτώνται από την βιβλιοθήκη τεχνολογίας. Όσο μεγαλύτερο το τρανζίστορ της τεχνολογίας, τόσο πιο αμελητέα η αντίσταση των καλωδίων καθώς και άλλων φυσικών παραμέτρων που επηρεάζουν την ταχύτητα. Η χρήση τους είναι κατάλληλη όταν οι περιορισμοί διαθέσιμων πόρων και κατανάλωση ενέργειας είναι ελαστικοί. 

