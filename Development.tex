\section{Ανάπτυξη Αθροιστών υπολοίπου $2^n-1$ }

Σε αυτό το κεφάλαιο θα αναπτυχθούν συνολικά δώδεκα αθροιστές υπολοίπου $2^n-1$
ακολουθώντας την αρχιτεκτονική που παρουσιάστηκε στο προηγούμενο κεφάλαιο 
με τα ελάχιστα επίπεδα. Ανάλογα με την παραγοντοποίηση που τους εφαρμόζεται 
οι αθροιστές ομαδοποιούνται σε τρείς ομάδες, Prefix, Ling και Jackson,
και σε κάθε ομάδα θα αναπτυχθεί ένας 8-bit, ένας 16-bit, ένας 32-bit και ένας 64-bit 
αθροιστής. Για κάθε ομάδα θα αναλύεται και σχηματικά ο 8-bit αθροιστής λόγω του 
ευδιάκριτου σχήματος που τον περιγράφει, ενώ για τους υπόλοιπους θα δοθεί η
συναρτησιακή λογική πλήρως σε άλγεβρα Μπουλ.

\subsection{Prefix $2^n-1$}
%\subsubsection{$2^8-1$}
%\subsubsection{$2^{16}-1$}
%\subsubsection{$2^{32}-1$}
%\subsubsection{$2^{64}-1$}


%---------------------------------------------------
\subsection{Ling $2^n-1$}
%---------------------------------------------------



%\subsubsection{$2^8-1$}
%%---------------------------------------------------
%\subsubsection{$2^{16}-1$}
%\subsubsection{$2^{32}-1$}
%\subsubsection{$2^{64}-1$}






%---------------------------------------------------
\subsection{Jackson $2^n-1$}
%---------------------------------------------------


\subsubsection{$2^8-1$}
%---------------------------------------------------

% Figure
%
%--------------------------------------------
\begin{figure}[ht]
\centering
\includegraphics[width=\textwidth]{J8_Color.png}
\caption{Jackson 8-bit $2^n-1$ Adder}
\label{Jackson_8-bit_Tree}
\end{figure}





Επίπεδο 1:\\
\begin{equation}
\begin{split}
p_i &= a_i + b_i\\
g_i &= a_i * b_i\\
x_i &= a_i \oplus b_i
\end{split}
\end{equation}
\\
Επίπεδο 2:\\
\begin{equation}
\begin{split}
R^1_{i:i-1} &= g_i + g_{i-1}\\
Q^1_{i:i-1} &= p_i * p_{i-1}\\
\end{split}
\end{equation}
\\
Επίπεδο 3:\\
\begin{equation}
\begin{split}
R^3_{i:i-7} =& R^1_{i:i-1} + R^1_{i-2:i-3} + Q^1_{i-3:i-4} R^1_{i-4:i-5} \\
            +& Q^1_{i-3:i-4} Q^1_{i-5:i-6} R^1_{i-6:i-7} 
\end{split}
\end{equation}
\\
Group Generate:\\
\begin{equation}
G_{i:i-7} = D_{i:i-2} R^3_{i:i-7}
\end{equation}
Όπου : 
\begin{equation}
\begin{split}
D_{i:i-2} &= G_{i:i-1} + P_{i:i-2}\\
D_{i:i-2} &= g_i + p_ig_{i-1} + p_ip_{i-1}p_{i-2}
\end{split}
\end{equation}
\\
Επίπεδο 5 - Sum computation:\\
\begin{equation}
% sum_i = !R^3_{i-1:i-8} * (a_i \oplus b_i) + R^3_{i-1:i-8} * (a_i \oplus b_i \oplus D_{i-1:i-3})
sum_i = R^3_{i-1:i-8} ? (x_i \oplus D_{i-1:i-3}) : x_i
\end{equation}






Για παράδειγμα:\\
\rule{\linewidth}{0.5mm}
\begin{equation*}
\begin{split}
p_7 =& a_7 + b_7\\
g_7 =& a_7 * b_7\\
R^1_{7:6} =& g_7 + g_{6}\\
Q^1_{7:6} =& p_7 * p_{6}\\
R^3_{7:0} =& R^1_{7:6} + R^1_{5:4} + Q^1_{4:3} R^1_{3:2} + Q^1_{4:3} Q^1_{2:1} R^1_{1:0}\\
D_{7:5} =& g_7 + p_7g_{6} + p_7p_{6}p_{5}\\
sum_7 =& !R^3_{6:7} * (a_7 \oplus b_7) + R^3_{6:7} * (a_7 \oplus b_7 \oplus D_{6:4})
\end{split}
\end{equation*}
\rule{\linewidth}{0.5mm}







%\subsubsection{$2^{16}-1$}
%%---------------------------------------------------
%Για παράδειγμα:\\
%\rule{\linewidth}{0.5mm}
%\begin{equation*}
%\begin{split}
%R^1_{15:12} =& g_{15} + g_{14} + p_{14}g_{13} + p_{14}p_{13}g_{12}\\
%Q^3_{15:12} =& p_{15} * p_{14} * p_{13} * p_{12}\\
%R^5_{15:0} =& R^1_{15:12} + R^1_{11:8} + Q^3_{10:7} R^1_{7:4} + Q^3_{10:7} Q^3_{6:3} R^1_{3:0}\\
%D_{15:11} =& p_{15}R^1_{15:12} + p_{11}Q^3_{15:12} \\
%sum_15 =& !R^5_{14:15} * (a_15 \oplus b_15) + R^5_{14:15} * (a_15 \oplus b_15 \oplus D_{14:10})
%\end{split}
%\end{equation*}
%\rule{\linewidth}{0.5mm}




%\subsubsection{$2^{32}-1$}
%%---------------------------------------------------
%
%
%
%
%
%\subsubsection{$2^{64}-1$}
%%---------------------------------------------------
%Για παράδειγμα:\\
%\rule{\linewidth}{0.5mm}
%\begin{equation*}
%\begin{split}
%R^1_{63:60} =& g_{63} + g_{62} + p_{62}g_{61} + p_{62}p_{61}g_{60}\\
%Q^3_{63:60} =& p_{63} * p_{62} * p_{61} * p_{60}\\
%R^5_{63:48} =& R^1_{63:60} + R^1_{59:56} + Q^3_{58:55} R^1_{55:52} + Q^3_{58:55} Q^3_{54:51} R^1_{51:48}\\
%Q^{11}_{63:48} =& Q^3_{63:60} Q^3_{59:56} Q^3_{55:52} ( R^1_{52:49} + Q^3_{51:48})\\
%R^{11}_{63:0} =& R^5_{63:48} + R^5_{47:32} + Q^{11}_{42:27} R^5_{31:16} + Q^{11}_{42:27} Q^{11}_{26:11} R^5_{15:0}\\
%D_{63:61} =& g_{63} + p_{63}g_{62} + p_{63}p_{62}p_{61}\\
%D_{63:59} =& D_{63:61} [R^1_{63:60} + Q^3_{62:59}]\\
%D_{63:43} =& D_{63:59} [R^5_{63:48} + Q^{11}_{58:43}]\\
%sum_63 =& !R^5_{62:63} * (a_63 \oplus b_63) + R^5_{62:63} * (a_63 \oplus b_63 \oplus D_{62:42})\\
%\end{split}
%\end{equation*}
%\rule{\linewidth}{0.5mm}

