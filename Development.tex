\section{Ανάπτυξη Αθροιστών υπολοίπου $2^n-1$ }

Σε αυτό το κεφάλαιο θα αναπτυχθούν συνολικά δώδεκα αθροιστές υπολοίπου $2^n-1$
ακολουθώντας την αρχιτεκτονική που παρουσιάστηκε στο προηγούμενο κεφάλαιο 
με τα ελάχιστα επίπεδα. Ανάλογα με το είδος της παραγοντοποίησης που τους εφαρμόζεται 
οι αθροιστές ομαδοποιούνται σε τρεις ομάδες, Prefix, Ling και Jackson,
και σε κάθε ομάδα θα αναπτυχθεί ένας 8-bit, ένας 16-bit, ένας 32-bit και ένας 64-bit 
αθροιστής. Κάθε αθροιστής που αναπτύσσεται περιγράφεται πλήρως από τις λογικές 
συναρτήσεις κάθε επιπέδου του. 

% Για κάθε ομάδα θα αναλύεται και σχηματικά ο 8-bit αθροιστής λόγω του 
% ευδιάκριτου σχήματος που τον περιγράφει, ενώ για τους υπόλοιπους θα δοθεί η
% συναρτησιακή λογική πλήρως σε άλγεβρα Μπουλ.




\subsection{Βασική δομή}
Στην εικόνα \ref{2^8-1_Tree_2x4} παρουσιάζεται προσεγγιστικά η δομή των αθροιστών υπολοίπου $2^8-1$ 
που θα χρησιμοποιηθεί παρακάτω. Ενώ η δομή είναι κοινή και για τα τρία είδη αθροιστών το κάθε σχήμα 
αντιπροσωπεύει διαφορετική λογική συνάρτηση. Επίσης στο δέντρο αυτό ακολουθείται η κατασκευή μόνο 
των σημάτων που οδηγούν τον τελευταίο πολυπλέκτη, δηλαδή στην περίπτωση του Prefix το σήμα G, στου Ling
το σήμα H και στο Jackson το R.
\begin{figure}[H]
\centering
\includegraphics[width=\textwidth]{J8_Color.png}
\caption{Jackson 8-bit $2^n-1$ Adder}
\label{2^8-1_Tree_2x4}
\end{figure}







\subsection{Prefix $2^n-1$}
%\subsubsection{$2^8-1$}
%\subsubsection{$2^{16}-1$}
%\subsubsection{$2^{32}-1$}
%\subsubsection{$2^{64}-1$}


%---------------------------------------------------
\subsection{Ling $2^n-1$}
%---------------------------------------------------



%\subsubsection{$2^8-1$}
%%---------------------------------------------------
%\subsubsection{$2^{16}-1$}
%\subsubsection{$2^{32}-1$}
%\subsubsection{$2^{64}-1$}






%---------------------------------------------------
\subsection{Jackson $2^n-1$}
%---------------------------------------------------
Στην ενότητα \ref{section:jackson} έγινε μια αναφορά σε σχεδιαστικούς κανόνες όσο αφορά τους 
αθροιστές Jackson. Για την ανάπτυξη ενός Jacson αθροιστή υπάρχει ένα μεγάλο πλήθος πιθανών υλοποιήσεων,
όχι μόνο στην επιλογή του προθεματικού δέντρου και το πλήθος των όρων της παραγοντοποίησης ( Radix-2, 
Radix-3, Radix-4 ... ), κάτι που αφορά και τους αθροιστές Prefix και Ling, αλλά 
και στην επιλογή της συνάρτησης παραγοντοποίησης σε κάθε επίπεδο.
\\
\textcolor{red}{[Γράψε όλους τους συνδυασμούς για έναν 8-bit ή 16-bit αθροιστή σε συντομία
π.χ. για 8 = 2x2x2, 2x4, 4x2 , και σε κάθε ένα τις πιθανές Burgess Συναρτήσεις]}\\
Όσο αφορά την επιλογή του πλήθους εισόδων στους κάβους του κάθε επίπεδου είναι επιθυμητό να αποφεύγεται 
η επιλογή του Radix-2 διότι, επαναδιατυπώνονταν, οι συναρτήσεις κατά Jacskon δεν διαφέρουν 
με αυτές ενός απλού Prefix. Θα ήταν αρκετά αποδοτική η υλοποίηση ορισμένων επιπέδων με Radix-3
, όπως και στους αθροιστές που αναπτύχθηκαν στα \cite{6189978} και \cite{6810474}, διότι
στις περιπτώσεις περισσότερων των δύο εισόδων έχει εφαρμογή η παραγοντοποίηση και επιπλέον, εφόσον 
οι κόμβοι θα είναι των τριών εισόδων αρά και οι συναρτήσεις πιο απλές, θα υπάρχουν μεγαλύτερες 
αντιστοιχίες των υλοποιημένων λογικών συναρτήσεων από τις CMOS βιβλιοθήκες. Στην περίπτωση των 
αθροιστών $2^n-1$, με $n=8, 16, 32$ και 64, δεν υπάρχει πολλαπλάσιο του τρία που δίνει τουλάχιστον έναν 
από αυτούς τους αριθμούς. Στην πραγματικότητα υπάρχει τρόπος αλλά το prefix-tree θα διαφέρει αρκετά με αυτό
του Kogge-stone και δεν θα υπάρχει το προνόμιο επαναχρησιμοποίησης κόμβων, με αποτέλεσμα η πολυπλοκότητα των
αθροιστών καθώς και η επιφάνεια και κατανάλωση να είναι αρκετά μεγαλύτερη.




\subsubsection{$2^8-1$}
%---------------------------------------------------

% Figure
%
%--------------------------------------------

\begin{table}[H]
\centering
     \begin{tabular}{ c  c  c  c  c } 
        \hline
        level & \multicolumn{4}{c}{R-Q Equations}\\
        \hline
        \hline
        0   & 
        \multicolumn{4}{c}{
        \begin{tabular}{@{}c@{}}$g_i = a_i * b_i$\\$p_i = a_i + b_i$\\$x_i = a_i \oplus b_i $\end{tabular}
        }
        
        \\ 
        \hline
        1 (x2)  & 
        \multicolumn{4}{c}{
        \begin{tabular}{@{}c@{}}$R1_i = g_i + g_{i-1}$\\$Q1_i = p_i * p_{i-1}$\end{tabular}
        }
       
        \\ 
        \hline
        2 (x4)  & 
        \multicolumn{4}{c}{
        \begin{tabular}{@{}c@{}}$R2_i = R1_i + R1_{i-2} + Q1_{i-3}*R1_{i-4} + Q1_{i-3}*Q1_{i-5}*R1_{i-6}$
        \end{tabular}
        }
        
        \\ 
        \hline
        D   & 
        \multicolumn{4}{c}{
        \begin{tabular}{@{}c@{}}$ D_i = g_i + p_ig_{i-1} + p_ip_{i-1}p_{i-2}$
        \end{tabular}
        }
        
        \\ 
        \hline
        SUM   & 
        \multicolumn{4}{c}{
        \begin{tabular}{@{}c@{}}$ sum_i = R2_{i-1}\ ?\ (x_i \oplus D_{i-1})\ :\ x_i$
        \end{tabular}
        }
        
        \\
        \hline
        \hline
    Symbols & $R1_i$ & $Q1_i$ & $R2_i$ & $D_i$ \\
    \hline
    Equations & $R^1_{i:i-1}$ & $Q^1_{i:i-1}$ & $R^3_{i:i-7}$ &$ D_{i:i-2}$ \\
    \hline
        
        
        \hline

    \end{tabular}
\caption{Jackson $2^{8}-1$ Equations}
\end{table}


\begin{table}[H]
\centering
    \begin{tabular}{|| c || c | c | c | c ||} 
    \hline
    Symbols & $R1_i$ & $Q1_i$ & $R2_i$ & $D_i$ \\
    \hline
    Equations & $R^1_{i:i-1}$ & $Q^1_{i:i-1}$ & $R^3_{i:i-7}$ &$ D_{i:i-2}$ \\
    \hline
    \end{tabular}
\end{table}     




% Επίπεδο 1:\\
% \begin{equation}
% \begin{split}
% p_i &= a_i + b_i\\
% g_i &= a_i * b_i\\
% x_i &= a_i \oplus b_i
% \end{split}
% \end{equation}
% \\
% Επίπεδο 2:\\
% \begin{equation}
% \begin{split}
% R^1_{i:i-1} &= g_i + g_{i-1}\\
% Q^1_{i:i-1} &= p_i * p_{i-1}\\
% \end{split}
% \end{equation}
% \\
% Επίπεδο 3:\\
% \begin{equation}
% \begin{split}
% R^3_{i:i-7} =& R^1_{i:i-1} + R^1_{i-2:i-3} + Q^1_{i-3:i-4} R^1_{i-4:i-5} \\
%             +& Q^1_{i-3:i-4} Q^1_{i-5:i-6} R^1_{i-6:i-7} 
% \end{split}
% \end{equation}
% \\
% Group Generate:\\
% \begin{equation}
% G_{i:i-7} = D_{i:i-2} R^3_{i:i-7}
% \end{equation}
% Όπου : 
% \begin{equation}
% \begin{split}
% D_{i:i-2} &= G_{i:i-1} + P_{i:i-2}\\
% D_{i:i-2} &= g_i + p_ig_{i-1} + p_ip_{i-1}p_{i-2}
% \end{split}
% \end{equation}
% \\
% Επίπεδο 5 - Sum computation:\\
% \begin{equation}
% % sum_i = !R^3_{i-1:i-8} * (a_i \oplus b_i) + R^3_{i-1:i-8} * (a_i \oplus b_i \oplus D_{i-1:i-3})
% sum_i = R^3_{i-1:i-8} ? (x_i \oplus D_{i-1:i-3}) : x_i
% \end{equation}






Για παράδειγμα:\\
\rule{\linewidth}{0.5mm}
\begin{equation*}
\begin{split}
p_7 =& a_7 + b_7\\
g_7 =& a_7 * b_7\\
x_7 =& a_7 \oplus b_7 \\
R^1_{7:6} =& g_7 + g_{6}\\
Q^1_{7:6} =& p_7 * p_{6}\\
R^3_{7:0} =& R^1_{7:6} + R^1_{5:4} + Q^1_{4:3} R^1_{3:2} + Q^1_{4:3} Q^1_{2:1} R^1_{1:0}\\
D_{7:5} =& g_7 + p_7g_{6} + p_7p_{6}p_{5}\\
sum_7 =& R^3_{6:7}\ ?\ x_7 \oplus D_{6:4}\ :\ x_7 
\end{split}
\end{equation*}
\rule{\linewidth}{0.5mm}











\subsubsection{Λογική περιγραφή για $n={16, 32, 64}$}











\begin{table}[H]
\centering
     \begin{tabular}{|| c | c ||}
     
        \hline
        level & R-Q Equations\\
        \hline
        \hline
        0   & 
        \begin{tabular}{@{}c@{}}$g_i = a_i * b_i$\\$p_i = a_i + b_i$\\$x_i = a_i \oplus b_i $\end{tabular}
        
        \\ 
        \hline
        1 (x4)  & 
        \begin{tabular}{@{}c@{}}
        $R1_i = g_i + g_{i-1} + p_{i-1}g_{i-2} + p_{i-1}p_{i-2}g_{i-3}$\\
        $Q1_i = p_ip_{i-1}p_{i-2}p_{i-3}$
        \end{tabular}
       
        \\ 
        \hline
        2 (x4)  & 
        \begin{tabular}{@{}c@{}}
        $R2_i = R1_i + R1_{i-4} + Q1_{i-5}*R1_{i-8} + Q1_{i-5}*Q1_{i-9}*R1_{i-12}$
        \end{tabular}
        
        \\ 
        \hline
        D   & 
        \begin{tabular}{@{}c@{}}$ D_i = p_iR1_i + p_{i-1}Q1_i$
        \end{tabular}
        
        \\ 
        \hline
        SUM   & 
        \begin{tabular}{@{}c@{}}$ sum_i = R2_{i-1}\ ?\ (x_i \oplus D_{i-1})\ :\ x_i$
        \end{tabular}
        
        \\
        \hline
    
    \end{tabular}
\caption{Jackson $2^{16}-1$ Equations}
\end{table}









\begin{table}[H]
\centering
     \begin{tabular}{|| c | c ||}
     
        \hline
        level & R-Q Equations\\
        \hline
        \hline
        0   & 
        \begin{tabular}{@{}c@{}}$g_i = a_i * b_i$\\$p_i = a_i + b_i$\\$x_i = a_i \oplus b_i $\end{tabular}
        
        \\ 
        \hline
        1 (x2)  & 
        \begin{tabular}{@{}c@{}}$R1_i = g_i + g_{i-1}$\\$Q1_i = p_i * p_{i-1}$\end{tabular}
       
        \\ 
        \hline
        2 (x4)  & 
        \begin{tabular}{@{}c@{}}
        $R2_i = R1_i + R1_{i-2} + Q1_{i-3}*R1_{i-4} + Q1_{i-3}*Q1_{i-5}*R1_{i-6}$\\
        $Q2_i = Q1_i Q1_{i-2} Q1_{i-4} ( R1_{i-5} + Q1_{i-6})$
        \end{tabular}
        
        \\ 
        \hline
        3 (x4)  & 
        \begin{tabular}{@{}c@{}}
        $R3_i = R2_i + R2_{i-8} + Q2_{i-11}*R2_{i-16} + Q2_{i-11}*Q2_{i-17}*R2_{i-24}$
        \end{tabular}
        
        \\ 
        \hline
        D   & 
        \begin{tabular}{@{}c@{}}$ D1_i = g_i + p_ig_{i-1} + p_ip_{i-1}p_{i-2}$\\
        $D_i = D1_i ( R2_i + Q2_{i-3} )$
        \end{tabular}
        
        \\ 
        \hline
        SUM   & 
        \begin{tabular}{@{}c@{}}$ sum_i = R2_{i-1}\ ?\ (x_i \oplus D_{i-1})\ :\ x_i$
        \end{tabular}
        
        \\
        \hline
    
    \end{tabular}
\caption{Jackson $2^{32}-1$ Equations}
\end{table}







%\subsubsection{$2^{16}-1$}
%%---------------------------------------------------
%Για παράδειγμα:\\
%\rule{\linewidth}{0.5mm}
%\begin{equation*}
%\begin{split}
%R^1_{15:12} =& g_{15} + g_{14} + p_{14}g_{13} + p_{14}p_{13}g_{12}\\
%Q^3_{15:12} =& p_{15} * p_{14} * p_{13} * p_{12}\\
%R^5_{15:0} =& R^1_{15:12} + R^1_{11:8} + Q^3_{10:7} R^1_{7:4} + Q^3_{10:7} Q^3_{6:3} R^1_{3:0}\\
%D_{15:11} =& p_{15}R^1_{15:12} + p_{11}Q^3_{15:12} \\
%sum_15 =& !R^5_{14:15} * (a_15 \oplus b_15) + R^5_{14:15} * (a_15 \oplus b_15 \oplus D_{14:10})
%\end{split}
%\end{equation*}
%\rule{\linewidth}{0.5mm}




%\subsubsection{$2^{32}-1$}
%%---------------------------------------------------
%
%
%
%
%
%\subsubsection{$2^{64}-1$}
%%---------------------------------------------------
%Για παράδειγμα:\\
%\rule{\linewidth}{0.5mm}
%\begin{equation*}
%\begin{split}
%R^1_{63:60} =& g_{63} + g_{62} + p_{62}g_{61} + p_{62}p_{61}g_{60}\\
%Q^3_{63:60} =& p_{63} * p_{62} * p_{61} * p_{60}\\
%R^5_{63:48} =& R^1_{63:60} + R^1_{59:56} + Q^3_{58:55} R^1_{55:52} + Q^3_{58:55} Q^3_{54:51} R^1_{51:48}\\
%Q^{11}_{63:48} =& Q^3_{63:60} Q^3_{59:56} Q^3_{55:52} ( R^1_{52:49} + Q^3_{51:48})\\
%R^{11}_{63:0} =& R^5_{63:48} + R^5_{47:32} + Q^{11}_{42:27} R^5_{31:16} + Q^{11}_{42:27} Q^{11}_{26:11} R^5_{15:0}\\
%D_{63:61} =& g_{63} + p_{63}g_{62} + p_{63}p_{62}p_{61}\\
%D_{63:59} =& D_{63:61} [R^1_{63:60} + Q^3_{62:59}]\\
%D_{63:43} =& D_{63:59} [R^5_{63:48} + Q^{11}_{58:43}]\\
%sum_63 =& !R^5_{62:63} * (a_63 \oplus b_63) + R^5_{62:63} * (a_63 \oplus b_63 \oplus D_{62:42})\\
%\end{split}
%\end{equation*}
%\rule{\linewidth}{0.5mm}












\subsection{Διαδικασία ελέγχου ορθής λειτουργίας}


